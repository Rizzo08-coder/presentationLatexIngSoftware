\subsection{Refactoring - estrazione dei controller MVC}
\beamertitle

\begin{frame}
    \frametitle{Refactor}
    Il refactor che abbiamo deciso di dettagliare consiste nell'estrazione di
    un controller MVC da una classe di view pre-refactor.
    \\
    \medskip
    In tutto sono stati utilizzati i seguenti pattern di refactor:
    \begin{itemize}
        \item \emph{Extract Class}
        \item \emph{Extract Superclass}
        \item \emph{Form Template Method}
        \item \emph{Extract Method}
        \item \emph{Move Method}
        \item \emph{Pull Up Method}
    \end{itemize}
\end{frame}

\begin{frame}
    \frametitle{Extract Class}
    \begin{enumerate}
        \item Rifattorizzato nome della View (es. \texttt{CategoryView} $\to$ \texttt{CategoryMVController})
        \item Creato una nuova classe View (es. \texttt{CategoryView})
        \item Referenziato la View all'interno del MVC Controller
        \item Spostati campi e metodi dal MVC controller alla View in base alle responsabilitá:
          \begin{itemize}
              \item \texttt{printHierarchies()}
              \item \texttt{askDescription()}
              \item \texttt{askRootName()}
              \item \texttt{askFields()}
              \item ...
          \end{itemize}
        \item Utilizzato \emph{Extract Method} e \emph{Move Method} per estrarre e spostare ulteriori responsabilitá proprie della View
              \\(es. \texttt{askFilePath()})
    \end{enumerate}
\end{frame}

\begin{frame}
    \frametitle{Extract SuperClass}
    \begin{enumerate}
        \item Creato la superClasse astratta \texttt{AbstractView}
        \item Fattorizzato i comportamenti comuni delle View in metodi concreti della superClasse
          \begin{itemize}
              \item \texttt{menu()}
              \item \texttt{message()}
              \item \texttt{selectOptionFromCollection()}
          \end{itemize}
        \item Creato nella superClasse dei metodi concreti di comodo per le View, che delegano ad una Chain of Responsibility la responsabilitá
              di fornire una rappresentazione testuale degli oggetti
        \item Fatto estendere le View dalla classe astratta \texttt{AbstractView} (es. \texttt{CategoryView extends AbstractView})
        \item Sostituite le chiamate a \texttt{System.out.println(...)} con il metodo \texttt{message()} di \texttt{AbstractView}
        \item Sostituite le chiamate implicite ai \texttt{toString()} delle classi di dominio con i metodi di \emph{render}
    \end{enumerate}
\end{frame}

\begin{frame}
    \frametitle{Form Template Method}
    \begin{enumerate}
        \item Creata una classe astratta \texttt{AbstractMVController}
        \item Applicato il pattern GoF \emph{Template Method} al metodo \texttt{execute()}
        \item Fatto \emph{Extract Method} sulle parti variabili del metodo \texttt{execute()} all'interno del Controller (conformi alla firma dei metodi di gancio di \emph{AbstractMVController})
        \item Fatto estendere i Controller MVC concreti dalla classe astratta \texttt{AbstractMVController}
    \end{enumerate}
\end{frame}

\begin{frame}
    \frametitle{Esempio di risultato con ArticleView e ArticleMVController}
    \begin{figure}
        \includegraphics[width=0.8\textwidth]{img/modello-vista_after.png}
    \end{figure}
\end{frame}
