\subsection{White-box}
\beamertitle
\begin{frame} [fragile]
    \frametitle{Cosa testare?}
    \begin{lstlisting}[autogobble, title={\texttt{Category.java}}]
       public Category searchTree(Strign catName) {
           if (this.name.equalsIgnoreCase(catName)) return this;
           for (Category child : children.values()) {
               Category lookup = child.searchTree(catName);
               if (lookup != null) return lookup;
           }
           return null;
       }
    \end{lstlisting}
    Perché proprio questo metodo?
    \begin{itemize}
        \item Rilevante nella logica di business
        \item Ricorsione aggiunge imprevedibilità
    \end{itemize}
    Metodologie di determinazione dei casi di test:
    \begin{itemize}
        \item \textbf{Cammini indipendenti}
        \item Per \textbf{cicli}
        \item Ulteriori casi di test aggiunti durante lo sviluppo
    \end{itemize}
\end{frame}

\begin{frame}
    \frametitle{Flow Graph}
    \begin{figure}
        \includegraphics[width=0.9\textwidth]{img/flowGraphSearchTree.png}
    \end{figure}
\end{frame}

\begin{frame}
    \frametitle{Complessità ciclomatica}
    La complessità ciclomatica può essere calcolata:
    \begin{enumerate}
        \item Contando le regioni: {\color{red}$4$}
        \item Contando archi e nodi: $12-10+2={\color{red}4}$
        \item Contando i nodi predicato: $3+1={\color{red}4}$
    \end{enumerate}
    Quindi $CC={\color{red}4}$.

    Cammini indipendenti:
    \begin{itemize}
        \item $1-9-10$
        \item $1-2-3-7-10$
        \item $1-2-3-4-5-8-10$
        \item $1-2-3-4-5-6-3-7-10$
    \end{itemize}
\end{frame}

\begin{frame}
    \frametitle{Casi di Test}
    \begin{figure}
        \includegraphics[width=0.8\textwidth]{img/casiTestWhiteBox.png}
    \end{figure}
    \begin{itemize}
        \item Testing per cammini indipendenti: $1, 2, 3, 4$
        \item Testing per cicli: $2, 7, 8$
        \begin{itemize}
            \item 2 $\to$ \texttt{n=0}
            \item 7 $\to$ \texttt{n=1}
            \item 8 $\to$ \texttt{n=2}
        \end{itemize}
        \item Test aggiuntivi: $5, 6$
    \end{itemize}
\end{frame}

\begin{frame} [fragile]
    \frametitle{Esempio di caso di test}
    \begin{lstlisting}[autogobble, title={\texttt{CategoryTest.java}}]
     @Test
     void searchTreeSuccessWithNoChildAndSameCategoryName() {
        Category category = new Category(
                "name",
                "description",
                Collections.emptyMap()
        );
        assertEquals(category, category.searchTree("name"));
     }
    \end{lstlisting}
    Con questo codice viene testato:
    \begin{itemize}
        \item Il cammino indipendente $1-9-10$
        \item Il ciclo eseguito con $0$ iterazioni
    \end{itemize}
\end{frame}

\subsection{Black box}
\beamertitle
\begin{frame}
    \frametitle{Testing funzionale}
    Per il testing Black box abbiamo verificato gli \emph{scenari principali dei casi d'uso
    più rilevanti}.
    In particolare ci è servito per fare \textbf{regression testing} durante l'estrazione
    degli MVC Controller.

    Per facilitare il testing abbiamo applicato \textbf{Dependency Inversion} ove necessario, così da poter
    usare dei \emph{mock} in fase di test dell'applicazione:
    \begin{itemize}
        \item \texttt{InMemoryDataContainer}: contenitore di dati non persistente.
        \item \texttt{QueueInputProvider}: provider di input non interattivo basato su code.
        \item \texttt{Clock.fixed(...)}: orologio costante per \emph{testing time-dependent}.
    \end{itemize}
\end{frame}

\begin{frame} [fragile]
    \frametitle{Il mock \texttt{QueueInputProvider}}
    \begin{lstlisting}[autogobble, title={\texttt{QueueInputProvider.java}}]
     private Queue<Integer> integerInputs;

     public void setIntegerInputs(Collection<Integer> integerInputs) {
        this.integerInputs = new LinkedList<>(integerInputs);
    }

     @Override
     public int leggiIntero(...) {
        return integerInputs.remove();
     }
    \end{lstlisting}
    \begin{lstlisting}[autogobble, title={\texttt{ArticleMVControllerTest.java}}]
    @Test
    void editArticle() {
      insertArticle();
      queueInputProvider.setIntegerInputs(List.of(4, 1, 1, 0));
      queueInputProvider.setBooleanInputs(List.of(true));
      articleMVController.execute(...);
      assertThat(out.toString(), containsString(...));
    }
    \end{lstlisting}
\end{frame}

\begin{frame}
    \frametitle{Altri test funzionali}
    Sono stati approntati casi di test anche per le funzionalità di:
    \begin{itemize}
        \item Salvataggio / caricamento su file tramite \emph{stream} di oggetti
        \item Caricamento da file .json in modalità \emph{batch}
    \end{itemize}
    Per rendere possibili questi test si sono usati:
    \begin{itemize}
        \item File temporanei, tramite \texttt{File.createTempFile(...)}
        \item Sovrascritture a \emph{test runtime} dei valori di \texttt{system.properties} con \texttt{System.getProperties().setProperty(...)}
    \end{itemize}
\end{frame}