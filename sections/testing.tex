\subsection{White-box}
\beamertitle
 \begin{frame} [fragile]
     \frametitle{Codice}
     \begin{lstlisting}[autogobble, title={\texttt{Category.java}}]
        public Category searchTree(Strign catName) {
            if (this.name.equalsIgnoreCase(catName)) return this;
            for (Category child : children.values()) {
                Category lookup = child.searchTree(catName);
                if (lookup != null) return lookup;
            }
            return null;
        }
     \end{lstlisting}
     \begin{itemize}
         \item Metodo che abbiamo deciso di testare siccome é fondamentale per la logica di business
         \item Contiene una ricorsione, quindi facile generare bug
         \item Utilizzato testing per cammini indipendenti:
           \begin{itemize}
               \item Determinati cammini indipendenti dopo aver costruito flow graph
               \item Fatti test per ogni cammino indipendente e ulteriori test aggiuntivi
           \end{itemize}
     \end{itemize}
 \end{frame}

 \begin{frame}
     \frametitle{Flow Graph}
     \begin{figure}
         \includegraphics[width=0.9\textwidth]{img/flowGraphSearchTree.png}
     \end{figure}
 \end{frame}

\begin{frame}
    \frametitle{Complessitá Ciclomatica}
    La complessitá ciclomatica puó essere calcolata:
    \begin{enumerate}
        \item Contando le regioni, ovvero $4$
        \item $12-10+2=4$, archi e nodi
        \item $3+1=4$, nodi predicato
    \end{enumerate}
    Dalla figura la complessitá ciclomatica é pari a 4

    \\
    Cammini indipendenti:
    \begin{itemize}
        \item $1-9-10$
        \item $1-2-3-7-10$
        \item $1-2-3-4-5-8-10$
        \item $1-2-3-4-5-6-3-7-10$
    \end{itemize}
\end{frame}

\subsection{Black-box}
\frame{\frametitle{Black-box}
    testing salvataggio
}