\subsection{Chain of Responsibility}
\beamertitle

\begin{frame}
    \frametitle{Chain of Responsibility}
    \begin{enumerate}
        \item Definita una \emph{Chain of Responsibility} per la conversione degli oggetti
        di dominio nelle loro rappresentazioni testuali.
        \item La \emph{Chain of Responsibility} è creata e gestita dalla \emph{View} astratta,
        che fornisce alle sottoclassi un metodo \texttt{render(...)}.
    \end{enumerate}

    \begin{figure}
        \includegraphics[width=0.8\textwidth]{img/chain.png}
    \end{figure}
\end{frame}

\begin{frame}[fragile]
  \frametitle{Uso}
  \begin{lstlisting}[autogobble, title={\texttt{AbstractView.java}}]
   private final AbstractRenderer chain = new ArticleRenderer(
                                 new CategoryRenderer(
                                 new ConfigRenderer(
                                 new ExchangeRenderer(
                                 new DefaultRenderer(null)))));

   protected String render(Object o) {
        return chain.render(o);
   }

   protected <T> String renderAll(Collection<T> collection) {
        return collection.stream().map(this::render)
                         .collect(Collectors.joining("\n"));
   }
  \end{lstlisting}
    \begin{itemize}
        \item \texttt{render(...)} è a disposizione delle \emph{View} concrete.
        \item \texttt{DefaultRenderer} fa da renderer \emph{catch-all}
        attraverso il \texttt{toString()} di \texttt{Object}.
    \end{itemize}
\end{frame}

\begin{frame}
    \frametitle{Esecuzione}
    \begin{figure}
        \includegraphics[width=1\textwidth]{img/sequenceChain.png}
        \caption{Esecuzione senza ottenere un risultato, percorrendo tutta la catena a vuoto

        Esecuzione con una categoria}
    \end{figure}
\end{frame}

\begin{frame}
    \frametitle{Miglioramento con Set Of Responsibility}

    \begin{figure}
        \includegraphics[width=0.7\textwidth]{img/setofresponsibility.png}
    \end{figure}
    \begin{itemize}
      \item Nessun renderer è ``più uguale degli altri''.
      \item Il caso \emph{catch-all} è definito dal creatore del set di renderer.
      \item La responsabilità di creare il set è di \texttt{SetRenderer}.
    \end{itemize}
\end{frame}


\subsection{Template Method}
\beamertitle

\begin{frame}
    \frametitle{Prima del refactor}
    Duplicazione di codice:
    \begin{itemize}
        \item Nelle \emph{View} (poi \emph{Controller MVC}), per la gestione del menu.
        \item Nelle classi di salvataggio.
    \end{itemize}
    Ma alcune parti sono variabili:
    \begin{center}
        \small
        \begin{tabular}{r|l}
            Classi \emph{View} / \emph{Controller} & \begin{tabular}{l}
                Voci del menu \\
                Azioni delle voci del menu \\
                Comportamento prima della visione del menu \\
            \end{tabular} \\
            \hline
            Classi di salvataggio & \begin{tabular}{l}
                Oggetto ``vuoto'' caricato di default \\
                Tipi diversi degli oggetti caricati / salvati \\
            \end{tabular}  \\
        \end{tabular}
    \end{center}
\end{frame}

\begin{frame}
    \frametitle{Soluzione}
    \begin{itemize}
        \item Usare i generici per i tipi diversi nelle classi di persistenza
        \item Usare \textbf{Template method} e metodi astratti di gancio per le parti variabili:
    \end{itemize}

    \begin{center}
        \small
        \begin{tabular}{r|l}
            Voci del menu & \texttt{Map<String, Runnable> getMenuOptions()} \\
            Azioni del menu & \texttt{Map<String, Runnable> getMenuOptions()} \\
            Logica prima del menu & \texttt{boolean beforeExecute()} \\
            \hline
            Istanza ``vuota'' di default & \texttt{T defaultSaveObject()} \\
        \end{tabular}
    \end{center}
\end{frame}

\begin{frame}
    \frametitle{Template Method}
    \begin{figure}
        \includegraphics[width=1\textwidth]{img/templateMethod.png}
    \end{figure}
\end{frame}

\begin{frame} [fragile]
    \frametitle{Controller MVC astratto}
    \begin{lstlisting}[autogobble, title={\texttt{AbstractMVController.java}}, morekeywords={beforeExecute, getView, getMenuOptions}]
    public void execute(User user) {
        if (!beforeExecute()) {
        //...
        }
        getView();
        //...
        getMenuOptions();
        //...
    }
    protected abstract boolean beforeExecute();
    protected abstract LinkedHashMap<String, Runnable> getMenuOptions(User user);
    protected abstract AbstractView getView();
    \end{lstlisting}
    \begin{itemize}
        \item \texttt{execute(...)} è un metodo Template definito e implementato da \emph{AbstractMVController}
        \item \texttt{beforeExecute()}, \texttt{getView()} e \texttt{getMenuOptions()} sono metodi di gancio
        implementati nelle classi figlie per personalizzare il comportamento della classe.
    \end{itemize}
\end{frame}

\begin{frame} [fragile]
 \frametitle{Controller MVC concreto}
   \begin{lstlisting}[autogobble, title={\texttt{ArticleMVController.java}}]
     @Override
     protected boolean beforeExecute() { return true; }

     @Override
     protected AbstractView getView() { return articleView; }

     @Override
     protected LinkedHashMap<String, Runnable> getMenuOptions(User user) {
         //implementazione specifica ...
     }
   \end{lstlisting}
    \begin{itemize}
        \item \texttt{beforeExecute()} ritorna \texttt{true} se deve continuare l'esecuzione
        e mostrare il menu, \texttt{false} altrimenti.
    \end{itemize}
\end{frame}