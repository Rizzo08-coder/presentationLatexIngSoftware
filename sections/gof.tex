\frame{\frametitle{Pattern GoF}
principi solid
}

\subsection{Chain of Responsibility}
\frame{\frametitle{Chain of Responsibility}
Without title somethink is missing.
}

\subsection{Template Method}
\begin{frame}
    \frametitle{Template Method}
    \begin{figure}
         
        \begin{multicols} {2}
            \includegraphics[width=0.50\textwidth]{img/templateMethodController.png}
            \columnbreak
            \includegraphics[width=0.50\textwidth]{img/templateMethodSave.png}
        \end{multicols}
    \end{figure}
\end{frame}

\begin{frame} [fragile]
    \frametitle{Uso}
    \begin{lstlisting}[autogobble, title={\texttt{AbstractMVController.java}}, morekeywords={getView, getMenuOptions}]
    public void execute(User user) {
        //...
        getView();
        //...
        getMenuOptions():
        //...
    }
    protected abstract LinkedHashMap<String, Runnable> getMenuOptions(User user);
    protected abstract AbstractView getView();
    \end{lstlisting}
    \begin{itemize}
        \item \texttt{execute(...)} é un metodo Template definito e implementato da \emph{AbstractMVController}
        \item \texttt{getMenuOptions()} e \texttt{getView()} metodi di gancio implementati nella classe figlia \emph{ArticleMVController} con un comportamento specifico
    \end{itemize}
\end{frame}

\begin{frame} [fragile]
    \frametitle{Uso}
    \begin{lstlisting}[autogobble, title={\texttt{ArticleMVController.java}}]
     @Override
     protected AbstractView getView() { return articleView; }

     @Override
     protected LinkedHashMap<String, Runnable> getMenuOptions(User user) {
         //implementazione specifica ...
     }
    \end{lstlisting}
\end{frame}
