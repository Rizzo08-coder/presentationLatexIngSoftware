\subsection{Single Responsibility}
\beamertitle

\frame{\frametitle{Single Responsibility - 1° caso}
    \begin{figure}
        \includegraphics[width=0.8\textwidth]{img/MV.png}
    \end{figure}
    Le classi dello strato di presentazione avevano 2 compiti:
    \begin{enumerate}
        \item Gestire le operazioni di input e di presentazione dei dati
        \item Dialogare con lo strato di dominio
    \end{enumerate}

    Soluzione: \textbf{estrarre i controller MVC dalle viste}.
}

\begin{frame}[fragile]
    \frametitle{Single Responsibility - 2° caso}
    \begin{lstlisting}[autogobble, title={\texttt{toString()} di \texttt{Article.java}}]
    @Override
    public String toString() {
        StringBuilder sb = new StringBuilder();
        sb.append(/* Articolo con ID, categoria, stato... */);
        for (Map.Entry<String, String> field:
              fields.entrySet()) {
            sb.append(/* Campo con chiave, valore */);
        }
        return sb.toString();
    }
    \end{lstlisting}

    Ogni classe dello stato di dominio ha 2 responsabilità:
    \begin{enumerate}
        \item Relativa alle operazioni di dominio per quella classe
        \item Relativa alla presentazione di quella classe (\texttt{toString()})
    \end{enumerate}

    Soluzione: \textbf{rimuovere i toString()} e \textbf{gestire la \emph{renderizzazione} dei dati di dominio
    attraverso delle classi dedicate con interfaccia comune}.
\end{frame}

\subsection{Dependency Inversion}
\beamertitle

\begin{frame}
    \frametitle{Dependency Inversion}
    Utilizzato in 3 casi diversi:
    \begin{itemize}
        \item Classe wrapper per la persistenza (\texttt{Saves}).
        \item Classe con metodi statici di input da CLI (\texttt{InputDati}).
        \item Classe orologio usata per lo scambio (\texttt{Clock}).
    \end{itemize}
    Vantaggi generali:
    \begin{itemize}
        \item {\color{green}Testing più facile} tramite oggetti \emph{mock}.
        \item {\color{green}Flessibilità} per sviluppi futuri.
    \end{itemize}
    Vediamo ora questi differenti casi più nel dettaglio.
\end{frame}

\begin{frame}
    \frametitle{DI - \texttt{Saves} $\to$ \texttt{DataContainer}}
    \texttt{Saves} era già stata oggetto di \emph{Dependency Injection} precedentemente al refactor.
    \textbf{Vantaggi}:
    \begin{itemize}
        \item {\color{green}Dipendenza esplicita} in ogni classe che la usa, che la richiede eventualmente sino al \texttt{Main}
        (entrypoint dell'applicazione).
        \item Spostamento delle implementazioni alla periferia del sistema, per una più {\color{green}facile configurabilità}.
    \end{itemize}

    Ma:
    \begin{itemize}
        \item Era una {\color{red}classe concreta}, per ottenere configurabilità serve che le classi \emph{dipendano da astrazioni}
        (\textbf{Dependency Inversion})
        \item I nomi dei file erano {\color{red}\emph{hardcoded}} all'interno di \texttt{Saves}.
    \end{itemize}
    \begin{figure}
        \includegraphics[width=0.8\textwidth]{img/dependencyInversion_before.png}
    \end{figure}
\end{frame}

\begin{frame}
    \frametitle{DI - \texttt{Saves} $\to$ \texttt{DataContainer}}
    \textbf{Soluzione}:
    \begin{itemize}
        \item Creazione di un'interfaccia \texttt{DataContainer} per applicare \emph{Dependency Inversion}.
        \item Uso di un file \texttt{system.properties} per definire i nomi dei file di salvataggio
        (così abbiamo \textbf{Protected Variation}).
    \end{itemize}
    Bisogna solo sostituire i parametri nei costruttori e le proprietà con la nuova astrazione.
    \begin{itemize}
        \item Per il testing automatico è stato usato un \texttt{mock} di \texttt{DataContainer} che non gestisce
        la persistenza: \texttt{InMemoryDataContainer}.
    \end{itemize}
    \begin{figure}
        \includegraphics[width=0.8\textwidth]{img/dependencyInversion_after.png}
    \end{figure}
\end{frame}

\begin{frame}
    \frametitle{DI - \texttt{InputDati} $\to$ \texttt{InputProvider}}
    \texttt{unibs.ing.fp.mylib.InputDati} è una classe di comodo con metodi statici per l'input
    da CLI, estesa di tanto in tanto con ulteriori metodi.

    \textbf{Problemi}:
    \begin{itemize}
        \item {\color{red}$\approx$ \textbf{Singleton}}, rendendo il testing molto difficile.
        \item E se non è più una CLI?
    \end{itemize}
\end{frame}

\begin{frame}
    \frametitle{DI - \texttt{InputDati} $\to$ \texttt{InputProvider}}
    \textbf{Soluzione}:
    \begin{enumerate}
        \item Trasformazione dei metodi statici di input in metodi d'istanza.
        \item Creazione di un'interfaccia attorno a \texttt{InputDati}: \texttt{InputProvider}.
    \end{enumerate}
    \begin{itemize}
        \item \texttt{InputProvider} viene impostato tramite \emph{Dependency Injection} a partire dal \emph{Main}
        ed è un attributo \texttt{protected} di \texttt{AbstractView}.
        \item Per il testing automatico è stato approntato un \emph{Mock} di \texttt{InputProvider} che usa delle code
        di valori da consumare progressivamente (\texttt{QueueInputProvider}).
    \end{itemize}
\end{frame}

\begin{frame}
    \frametitle{DI - \texttt{Clock}}
    \textbf{Problema}:
    \begin{itemize}
        \item Il testing automatico del flusso di scambio è {\color{red}\emph{time-dependent}}.
    \end{itemize}
    \textbf{Soluzione}:
    \begin{itemize}
        \item Derivare l'ora in \texttt{Exchange} a partire da un'implementazione di \texttt{Clock}
        fornita tramite \emph{Dependency Injection}.
    \end{itemize}
    \textbf{Vantaggi}:
    \begin{itemize}
        \item La configurazione dell'orologio usato dall'applicazione è alla sua periferia
        ({\color{green}facile configurazione}, ad es. del fuso orario tramite \texttt{system.properties}).
        \item {\color{green}Il testing automatico diventa deterministico} fornendo un orologio fisso con \texttt{Clock.fixed(...)}
    \end{itemize}
\end{frame}