\subsection{Single Responsibility}
\frame{\frametitle{Single Responsibility - 1° caso}
    \begin{figure}
        \includegraphics[width=0.8\textwidth]{img/MV.png}
    \end{figure}
    Le classi dello strato di presentazione avevano 2 compiti:
    \begin{enumerate}
        \item Gestire le operazioni di input e di presentazione dei dati
        \item Dialogare con lo strato di dominio
    \end{enumerate}

    Soluzione: \textbf{estrarre i controller MVC dalle viste}.
}

\begin{frame}[fragile]
    \frametitle{Single Responsibility - 2° caso}
    \begin{lstlisting}[autogobble, title={\texttt{toString()} di \texttt{Article.java}}]
    @Override
    public String toString() {
        StringBuilder sb = new StringBuilder();
        sb.append(/* Articolo con ID, categoria, stato... */);
        for (Map.Entry<String, String> field:
              fields.entrySet()) {
            sb.append(/* Campo con chiave, valore */);
        }
        return sb.toString();
    }
    \end{lstlisting}

    Ogni classe dello stato di dominio ha 2 responsabilità:
    \begin{enumerate}
        \item Relativa alle operazioni di dominio per quella classe
        \item Relativa alla presentazione di quella classe (\texttt{toString()})
    \end{enumerate}

    Soluzione: \textbf{rimuovere i toString()} e \textbf{gestire la \emph{renderizzazione} dei dati di dominio
    attraverso delle classi dedicate con interfaccia comune}.
\end{frame}

\subsection{Dependency Inversion}
\frame{\frametitle{Dependency Inversion}
    Without title somethink is missing.
}