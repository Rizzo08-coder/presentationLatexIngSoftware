\subsection{GRASP controller}
\beamertitle

\begin{frame}
    \frametitle{Pattern GRASP}
    \begin{figure}
        \includegraphics[width=1\textwidth]{img/MV.png}
    \end{figure}
    \begin{itemize}
        \item Si puó notare che giá dalla prima versione é stato implementato un GRASP Controller di tipo \emph{Facade}
        \item Rappresenta il dominio complessivo di \emph{Article}
    \end{itemize}
\end{frame}

\begin{frame}
    \frametitle{Pattern GRASP}
    \begin{figure}
        \includegraphics[width=1\textwidth]{img/sequenzaGraspController.png}
    \end{figure}
    \begin{itemize}
        \item \emph{ArticleController} é il primo oggetto che riceve richieste dalla UI (\emph{ArticleView}) e interagisce con il dominio
    \end{itemize}
\end{frame}




\subsection{Information Expert}
\beamertitle

\begin{frame} [fragile]
  \frametitle{Information Expert}
    \begin{lstlisting}[autogobble, title={\texttt{CategoryController.java}}]
    private Category searchTree(Category category, String name){
      for (var child : category.getChildren().entrySet()) {
        if (child.getValue().getName().equalsIgnoreCase(name)) {
          return child.getValue();
        } else if (!child.getValue().isLeaf()) {
          if (searchTree(child.getValue(), name) != null) {
            return searchTree(child.getValue(), name);
          }
        }
      }
      return null;
    }
    \end{lstlisting}
  \begin{itemize}
    \item L'assegnazione della responsabilità di ricercare all'interno dell'albero delle categorie
    secondo \textbf{Information Expert} andrebbe assegnata alla classe \texttt{Category}, per ridurre
    l'accoppiamento e garantire l'\emph{incapsulamento}.
    \item Così non è: la responsabilità è di \texttt{CategoryController}.
    \item \emph{Code smell}: il codice utilizza molti \emph{getter} di \texttt{Category}.
  \end{itemize}
\end{frame}

\begin{frame} [fragile]
    \frametitle{Information Expert}
    \begin{lstlisting}[autogobble, title={\texttt{Category.java}}]
        public Category searchTree(Strign catName) {
            if (this.name.equalsIgnoreCase(catName)) return this;
            for (Category child : children.values()) {
                Category lookup = child.searchTree(catName);
                if (lookup != null) return lookup;
            }
            return null;
        }
    \end{lstlisting}
    Vantaggi:
    \begin{itemize}
        \item Sono diminuiti i parametri del metodo: {\color{green}minore accoppiamento} e {\color{green}migliore comprensibilità}.
        \item Non abbiamo piú tutti i \emph{getter} di prima: {\color{green}miglior incapsulamento}.
    \end{itemize}
    All'interno di \texttt{CategoryController} rimane una delega al metodo \texttt{searchTree(...)} di \texttt{Category}.
\end{frame}
