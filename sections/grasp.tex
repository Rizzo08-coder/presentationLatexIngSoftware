\subsection{GRASP controller}
\begin{frame}
    \frametitle{Pattern GRASP}
    \begin{figure}
        \includegraphics[width=1\textwidth]{img/MV.png}
    \end{figure}
    \begin{itemize}
        \item Si puó notare che giá dalla prima versione é stato implementato un GRASP Controller di tipo \emph{Facade}
        \item Rappresenta il dominio complessivo di \emph{Article}
    \end{itemize}
\end{frame}

\begin{frame}
    \frametitle{Pattern GRASP}
    \begin{figure}
        \includegraphics[width=1\textwidth]{img/sequenzaGraspController_before.png}
    \end{figure}
    \begin{itemize}
        \item É il primo oggetto che riceve richieste dalla UI (\emph{ArticleView}) e interagisce con il dominio
    \end{itemize}
\end{frame}

\begin{frame}
    \frametitle{Soluzione Migliorata}

\end{frame}



\subsection{Information Expert}
    \begin{frame} [fragile]
      \frametitle{Information Expert}
        \begin{lstlisting}[autogobble, title={\texttt{CategoryController.java}}]
            private Category searchTree(Category category, String name) {
        for (Map.Entry<String, Category> child : category.getChildren().entrySet()) {
            if (child.getValue().getName().equalsIgnoreCase(name)) {
                return child.getValue();
            } else if (!child.getValue().isLeaf()) {
                if (searchTree(child.getValue(), name) != null) {
                    return searchTree(child.getValue(), name);
                }
            }
        }
        return null;
      }
        \end{lstlisting}
      \begin{itemize}
          \item Questo codice viola \emph{Information Expert} perché
          viene passato come parametro l'intera classe \emph{Category}, quindi sarebbe meglio
          mettere le responsabilitá dove sono contenuti i dati, ovvero in \emph{Category}
          \item Lo si nota anche dai molti \texttt{get...()} presenti
      \end{itemize}
    \end{frame}

\begin{frame} [fragile]
    \frametitle{Information Expert}
    \begin{lstlisting}[autogobble, title={\texttt{Category.java}}]
        public Category searchTree(String categoryName) {
        if (this.name.equalsIgnoreCase(categoryName)) return this;
        if (isLeaf()) return null;
        for (Category child : children.values()) {
            Category lookup = child.searchTree(categoryName);
            if (lookup != null) return lookup;
        }
        return null;
    }
    \end{lstlisting}
    \begin{itemize}
        \item Si puó notare che sono diminuiti i parametri del metodo, questo aiuta anche la comprensione dello stesso
        \item Non abbiamo piú tutti i \texttt{get...()} che utilizzavamo prima
    \end{itemize}
\end{frame}
